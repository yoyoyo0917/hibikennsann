\documentclass[a4paper, 11pt]{ltjarticle}
\usepackage{enumerate}
\usepackage{tikz}
\usepackage{graphicx}
\usepackage[version=3]{mhchem}
\usepackage{fancyhdr}
\usepackage{lastpage}
\usepackage{adjustbox}
\usepackage[bookmarks=true, hidelinks]{hyperref}

\newcommand{\seq}[1]{\begin{equation}
  \begin{split}
    #1
  \end{split}
\end{equation}}
\newcommand{\eq}[1]{\begin{align}
  #1
\end{align}
}
\newcommand{\longeq}[1]{\begin{multline}
  #1
\end{multline}
}
\newcommand{\f}[2]{\frac{#1}{#2}}
\newcommand{\pardiff}[2]{\f{\partial{#1}}{\partial{#2}}}
\newcommand{\const}[2][ ]{\mathrm{#2}_{\mathrm{#1}}}
\begin{document}
\begin{enumerate}
  \item 等方的でない2次元調和振動子を考え、振動数$\nu_1 \ll \nu_2$とする。このとき、比熱$C_v$に関する以下の式を導け。
  \eq{
    T \simeq T_1: C_v &= k_B\{(\f{h\nu_1}{k_BT})^2e^{-\f{h\nu_1}{k_BT}} + \dots\} \\
    T \simeq T_2: C_v &= k_B\{1 - \f{1}{12}(\f{h\nu_1}{k_BT})^2 + (\f{h\nu_2}{k_BT})^2e^{-\f{h\nu_2}{k_BT}} + \dots\} \\
    T \simeq T_3: C_v &= k_B\{2-\f{1}{21}(\f{h\nu_2}{k_BT})^2 + \dots\}
  }
  ただし、$k_BT_1 \ll h\nu_1\ll k_BT_2 \ll h\nu_2 \ll k_BT_3$とする。
  \\\\\\
  解答\\
  \hrulefill\\
  
  平均エネルギーは、ヘルムホルツの自由エネルギー$F$を用いて、
  \eq{
    \bar{\epsilon} = \f{\partial \f{F}{T}}{\partial \f{1}{T}}
  }
  と表される。2次元の調和振動子の分配関数$Z$は、
  \eq{\label{eq:Z2DHO}
    Z &= \f{e}{N}\sum_{n_1,n_2} e^{-\f{\epsilon_{n_1,n_2}}{k_BT}} \\
    &= e^{-\f{F}{kT}}
  }
  であるから、平均エネルギーは、
  \eq{
    \bar{\epsilon} &= \pardiff{ }{\f{1}{T}}(-k_B \ln Z) \\
    &= k_B T^2 \pardiff{\ln Z}{T}
  }
  式\eqref{eq:Z2DHO}において、
  \eq{
    \epsilon_{n_1,n_2} = (n_1 + \f{1}{2})h\nu_1 + (n_2 + \f{1}{2})h\nu_2
  }
  であるから、等比級数の公式等を用いて、
  \eq{
    Z = \f{e}{N} e^{-\f{h(\nu_1 + \nu_2)}{2k_BT}}(1 - e^{-\f{h\nu_1}{k_BT}})^{-1}(1 - e^{-\f{h\nu_2}{k_BT}})^{-1}
  }
  これで、自然対数をとり、$T$で微分すると、
  \eq{
    \bar{\epsilon} = \f{1}{2}h(\nu_1 + \nu_2) + \f{h\nu_1}{e^{\f{h\nu_1}{k_BT}} - 1} + \f{h\nu_2}{e^{\f{h\nu_2}{k_BT}} - 1}
  }
  ここで、右辺第2,3項はテイラー展開不可なのでローラン展開を行う。\\
  ローラン展開
  \eq{
    \f{1}{e^x - 1} &= \sum_{n=0}^{\infty} \f{B_n}{n!} x^{n-1} \\
    &= \f{1}{x} - \f{1}{2} + \f{x}{12} - \f{x^3}{720} + \dots
  }
  ただし、$B_n$はベルヌーイ数である。
  これにより、
  \eq{
    T \simeq T_1: \bar{\epsilon} &= \f{1}{2}h(\nu_1 + \nu_2) + h\nu_1 e^{-\f{h\nu_1}{k_BT}} + h\nu_2 e^{-\f{h\nu_2}{k_BT}} \\
    T \simeq T_2: \bar{\epsilon} &= \f{1}{2}h\nu_2 + k_BT + h\nu_2 e^{-\f{h\nu_2}{k_BT}} + \f{1}{12}\f{(h\nu_1)^2}{k_BT} \\
    T \simeq T_3: \bar{\epsilon} &= 2k_BT + \f{1}{12}\f{h^2(\nu_1^2 + \nu_2^2)}{k_BT}
  }
  となる。比熱は、
  \eq{
    C_v = \pardiff{\bar{\epsilon}}{T}
  }
  で与えられるから、求める式が導かれる。
\end{enumerate}
\nocite{*}\bibliography{bib/bibtest.bib}
\bibliographystyle{cv}
\end{document}