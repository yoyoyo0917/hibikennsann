\documentclass[a4paper, 11pt]{ltjarticle}
\usepackage{enumerate}
\usepackage{tikz}
\usepackage{graphicx}
\usepackage[version=3]{mhchem}
\usepackage{fancyhdr}
\usepackage{lastpage}
\usepackage{adjustbox}
\usepackage{physics}
\usepackage[bookmarks=true, hidelinks]{hyperref}

\newcommand{\seq}[1]{\begin{equation}
  \begin{split}
    #1
  \end{split}
\end{equation}}
\newcommand{\eq}[1]{\begin{align}
  #1
\end{align}
}
\newcommand{\longeq}[1]{\begin{multline}
  #1
\end{multline}
}
\newcommand{\f}[2]{\frac{#1}{#2}}
\newcommand{\pardiff}[2]{\f{\partial{#1}}{\partial{#2}}}
\newcommand{\const}[2][ ]{\mathrm{#2}_{\mathrm{#1}}}
\begin{document}
\begin{enumerate}
  \item 等方的でない2次元調和振動子を考え、振動数$\nu_1 \ll \nu_2$とする。このとき、比熱$C_v$に関する以下の式を導け。
  \eq{
    T \simeq T_1: C_v &= k_B\{(\f{h\nu_1}{k_BT})^2e^{-\f{h\nu_1}{k_BT}} + \dots\} \\
    T \simeq T_2: C_v &= k_B\{1 - \f{1}{12}(\f{h\nu_1}{k_BT})^2 + (\f{h\nu_2}{k_BT})^2e^{-\f{h\nu_2}{k_BT}} + \dots\} \\
    T \simeq T_3: C_v &= k_B\{2-\f{1}{21}(\f{h\nu_2}{k_BT})^2 + \dots\}
  }
  ただし、$k_BT_1 \ll h\nu_1\ll k_BT_2 \ll h\nu_2 \ll k_BT_3$とする。
  \\\\\\
  解答\\
  \hrulefill\\
  
  平均エネルギーは、ヘルムホルツの自由エネルギー$F$を用いて、
  \eq{
    \bar{\epsilon} = \f{\partial \f{F}{T}}{\partial \f{1}{T}}
  }
  と表される。2次元の調和振動子の分配関数$Z$は、
  \eq{\label{eq:Z2DHO}
    Z &= \f{e}{N}\sum_{n_1,n_2} e^{-\f{\epsilon_{n_1,n_2}}{k_BT}} \\
    &= e^{-\f{F}{kT}}
  }
  であるから、平均エネルギーは、
  \eq{
    \bar{\epsilon} &= \pardiff{ }{\f{1}{T}}(-k_B \ln Z) \\
    &= k_B T^2 \pardiff{\ln Z}{T}
  }
  式\eqref{eq:Z2DHO}において、
  \eq{
    \epsilon_{n_1,n_2} = (n_1 + \f{1}{2})h\nu_1 + (n_2 + \f{1}{2})h\nu_2
  }
  であるから、等比級数の公式等を用いて、
  \eq{
    Z = \f{e}{N} e^{-\f{h(\nu_1 + \nu_2)}{2k_BT}}(1 - e^{-\f{h\nu_1}{k_BT}})^{-1}(1 - e^{-\f{h\nu_2}{k_BT}})^{-1}
  }
  これで、自然対数をとり、$T$で微分すると、
  \eq{
    \bar{\epsilon} = \f{1}{2}h(\nu_1 + \nu_2) + \f{h\nu_1}{e^{\f{h\nu_1}{k_BT}} - 1} + \f{h\nu_2}{e^{\f{h\nu_2}{k_BT}} - 1}
  }
  ここで、右辺第2,3項はベルヌーイ数の定義であるテイラー展開を行う。\\
  ベルヌーイ数$B_n$の定義$\f{x}{e^x - 1}$のテイラー展開からxを除したもので、
  \eq{
    \f{1}{e^x - 1} &= \sum_{n=0}^{\infty} \f{B_n}{n!} x^{n-1} \\
    &= \f{1}{x} - \f{1}{2} + \f{x}{12} - \f{x^3}{720} + \dots
  }
  これにより、
  \eq{
    T \simeq T_1: \bar{\epsilon} &= \f{1}{2}h(\nu_1 + \nu_2) + h\nu_1 e^{-\f{h\nu_1}{k_BT}} + h\nu_2 e^{-\f{h\nu_2}{k_BT}} \\
    T \simeq T_2: \bar{\epsilon} &= \f{1}{2}h\nu_2 + k_BT + h\nu_2 e^{-\f{h\nu_2}{k_BT}} + \f{1}{12}\f{(h\nu_1)^2}{k_BT} \\
    T \simeq T_3: \bar{\epsilon} &= 2k_BT + \f{1}{12}\f{h^2(\nu_1^2 + \nu_2^2)}{k_BT}
  }
  となる。比熱は、
  \eq{
    C_v = \pardiff{\bar{\epsilon}}{T}
  }
  で与えられるから、求める式が導かれる。\\
  この式は、高温で古典的な結果$C_v = 2k_B$に近づき、低温で自由度が凍結することを示している。
  \\\\\\
  \item 時間に依存しない摂動論について、ハミルトニアンに微小な摂動$\hat{V}$が加わった時、波動関数の1次の補正項$\psi_n^{(1)}$は以下の式で与えられることを示せ。
  \eq{
    \psi_n^{(1)} = \sum_{m \neq n} \f{V_{mn}}{E_n^{(0)} - E_m^{(0)}} \psi_m^{(0)}
  }
  \\\\\\
  解答\\
  \hrulefill\\
  基本的に、次数の比較がメインである。\\
  まず、摂動を考慮したシュレディンガー方程式は、
  \eq{
    (\hat{H}_0 + \hat{V})\psi = E \psi\label{eq:perturbationSchrodinger}
  }
  である。\\
  波動関数$\psi$を0次の波動関数$\psi_n^{(0)}$で展開すると、
  \eq{
    \psi = \sum_k c_k \psi_k^{(0)}\label{eq:perturbationWavefunction}
  }
  となる。これをシュレディンガー方程式に代入すると、
  \eq{
    \sum_k c_k (\hat{H}_0 + \hat{V}) \psi_k^{(0)} = E \sum_k c_k \psi_k^{(0)}
  }
  となる。\\
  ここで、$\psi_m^{(0)}$と内積を取る($\bra{\psi_m^{(0)}}$を左から作用させる)と、
  \eq{
    c_m E_m^{(0)} + \sum_k c_k V_{mk} = E c_m\label{eq:perturbation1}
  }
  となる。ただし、\\
  $E_n=E_n^{(0)} + E_n^{(1)} + \dots$、\\
  $c_k = c_k^{(0)} + c_k^{(1)} + \dots$と表記していくことにする。\\
  (今回は1次までしか使わないが\dots)\\
  式\eqref{eq:perturbation1}を整理すると、、
  \eq{
    (E-E_m^{(0)}) c_m = \sum_k c_k V_{mk}\label{eq:perturbation2}
  }
  となる。注意すべきは、Eに添え字がないことである。\\
  これは、式\eqref{eq:perturbationSchrodinger}が一般的なシュレディンガー方程式であり、特定の固有状態に対応していないためである。\\
  左辺は演算子が代入前に残っていたため、特定の固有状態を記述できたが、右辺は演算子が消費(?)されてしまった後の式であるから、代入後には特定の固有状態を記述できない。\\
  ここから、状態$n$に関して考え、$E=E_n$とする。\\
  固有状態$n$の下では、波動関数の確率的解釈より$c_n^{(0)}=1$、$c_k^{(1)}=0$($k \neq n$)でないといけない。\\
  式で追うと、式\eqref{eq:perturbationWavefunction}において左辺の0次の項が$\psi_n^{(0)}$になるためである。($E=E_n$は$\psi=\psi_n$をもたらした。逆もまた然り)\\
  そうすると、式\eqref{eq:perturbation2}は、
  \eq{
    (E_n - E_m^{(0)}) c_m = V_{mn}\label{eq:perturbation3}
  }
  となる。\\
  ここから、$m=n$のときと$m \neq n$のときに分けて考える。\\
  まず、$m=n$のとき、式\eqref{eq:perturbation3}は、
  \eq{
    E_n^{(1)} = V_{nn}\\
    \forall c_n^{(1)} 
  }
  となる。\\
  次に、$m \neq n$のとき、式\eqref{eq:perturbation3}は、
  \eq{
    c_m^{(1)} = \f{V_{mn}}{E_n^{(0)} - E_m^{(0)}}
  }
  となる。\\
  $c_n^{(1)}$は任意であるため、規格化条件を満たすように$0$とした。\\
  以上より、波動関数の1次の補正項は、
  \eq{
    \psi_n^{(1)} = \sum_{m \neq n} c_m^{(1)} \psi_m^{(0)} = \sum_{m \neq n} \f{V_{mn}}{E_n^{(0)} - E_m^{(0)}} \psi_m^{(0)}
  }
  となる。これで示された。\\
  規格化について補足すると、$\abs{\psi_n}=1$を満たすためには、1次の補正項に関しては$\bra{\psi_n^{(0)}}\ket{\psi_n^{(1)}} + \bra{\psi_n^{(1)}}\ket{\psi_n^{(0)}} = 0$を満たす必要がある。($\bra{\psi_n^{(1)}}\ket{\psi_n^{(1)}}$は2次の微少量
  であり、$\bra{\psi_n^{(0)}}\ket{\psi_n^{(0)}}=1$のため)\\
  これは、$c_n^{(1)}$を任意に選べることから、$c_n^{(1)}=0$とすることで残りは直交条件から満たされる。\\
\end{enumerate}
\nocite{*}\bibliography{bib/bibtest.bib}
\bibliographystyle{cv}
\end{document}