\item 時間に依存しない摂動論について、ハミルトニアンに微小な摂動$\hat{V}$が加わった時、波動関数の1次の補正項$\psi_n^{(1)}$は以下の式で与えられることを示せ。
  \eq{
    \psi_n^{(1)} = \sum_{m \neq n} \f{V_{mn}}{E_n^{(0)} - E_m^{(0)}} \psi_m^{(0)}
  }
  \\\\\\
  解答\\
  \hrulefill\\
  基本的に、次数の比較がメインである。\\
  まず、摂動を考慮したシュレディンガー方程式は、
  \eq{
    (\hat{H}_0 + \hat{V})\psi = E \psi\label{eq:perturbationSchrodinger}
  }
  である。\\
  波動関数$\psi$を0次の波動関数$\psi_n^{(0)}$で展開すると、
  \eq{
    \psi = \sum_k c_k \psi_k^{(0)}\label{eq:perturbationWavefunction}
  }
  となる。これをシュレディンガー方程式に代入すると、
  \eq{
    \sum_k c_k (\hat{H}_0 + \hat{V}) \psi_k^{(0)} = E \sum_k c_k \psi_k^{(0)}
  }
  となる。\\
  ここで、$\psi_m^{(0)}$と内積を取る($\bra{\psi_m^{(0)}}$を左から作用させる)と、
  \eq{
    c_m E_m^{(0)} + \sum_k c_k V_{mk} = E c_m\label{eq:perturbation1}
  }
  となる。ただし、\\
  $E_n=E_n^{(0)} + E_n^{(1)} + \dots$、\\
  $c_k = c_k^{(0)} + c_k^{(1)} + \dots$と表記していくことにする。\\
  (今回は1次までしか使わないが\dots)\\
  式\eqref{eq:perturbation1}を整理すると、、
  \eq{
    (E-E_m^{(0)}) c_m = \sum_k c_k V_{mk}\label{eq:perturbation2}
  }
  となる。注意すべきは、Eに添え字がないことである。\\
  これは、式\eqref{eq:perturbationSchrodinger}が一般的なシュレディンガー方程式であり、特定の固有状態に対応していないためである。\\
  左辺は演算子が代入前に残っていたため、特定の固有状態を記述できたが、右辺は演算子が消費(?)されてしまった後の式であるから、代入後には特定の固有状態を記述できない。\\
  ここから、状態$n$に関して考え、$E=E_n$とする。\\
  固有状態$n$の下では、波動関数の確率的解釈より$c_n^{(0)}=1$、$c_k^{(1)}=0$($k \neq n$)でないといけない。\\
  式で追うと、式\eqref{eq:perturbationWavefunction}において左辺の0次の項が$\psi_n^{(0)}$になるためである。($E=E_n$は$\psi=\psi_n$をもたらした。逆もまた然り)\\
  そうすると、式\eqref{eq:perturbation2}は、
  \eq{
    (E_n - E_m^{(0)}) c_m = V_{mn}\label{eq:perturbation3}
  }
  となる。\\
  ここから、$m=n$のときと$m \neq n$のときに分けて考える。\\
  まず、$m=n$のとき、式\eqref{eq:perturbation3}は、
  \eq{
    E_n^{(1)} = V_{nn}\\
    \forall c_n^{(1)} 
  }
  となる。\\
  $E_n^{(0)}-E_m^{(0)}=0$より、$c_n^{(1)}$は任意である。\\
  次に、$m \neq n$のとき、式\eqref{eq:perturbation3}は、
  \eq{
    c_m^{(1)} = \f{V_{mn}}{E_n^{(0)} - E_m^{(0)}}
  }
  となる。\\
  $c_n^{(1)}$は任意であるため、規格化条件を満たすように$0$とした。\\
  以上より、波動関数の1次の補正項は、
  \eq{
    \psi_n^{(1)} = \sum_{m \neq n} c_m^{(1)} \psi_m^{(0)} = \sum_{m \neq n} \f{V_{mn}}{E_n^{(0)} - E_m^{(0)}} \psi_m^{(0)}
  }
  となる。これで示された。\\
  規格化について補足すると、$\abs{\psi_n}=1$を満たすためには、1次の補正項に関しては$\bra{\psi_n^{(0)}}\ket{\psi_n^{(1)}} + \bra{\psi_n^{(1)}}\ket{\psi_n^{(0)}} = 0$を満たす必要がある。($\bra{\psi_n^{(1)}}\ket{\psi_n^{(1)}}$は2次の微少量
  であり、$\bra{\psi_n^{(0)}}\ket{\psi_n^{(0)}}=1$のため)\\
  これは、$c_n^{(1)}$を任意に選べることから、$c_n^{(1)}=0$とすることで残りは直交条件から満たされる。\\