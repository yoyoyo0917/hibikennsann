\item 光学から類比して古典力学から量子力学に移行せよ。(ヒント: 幾何光学から波動光学への移行を参考にせよ)
  \\\\\\
  解答\\
  \hrulefill\\
  幾何光学にはフェルマーの原理が存在する。\\
  フェルマーの原理は、光が通る経路は光が通りうる経路の中で光が通るのに要する時間が最小となる経路である。つまり、光のはじめと終わりの位相差が最小となるような経路を選択する。\\
  古典力学において、最小作用の原理は、力学系が通る経路は通りうる経路の中で作用が最小となる経路である。\\
  ここで、作用$S$と位相$\phi$に類似性があることに注目する。\\
  $S \propto \phi$と考える。\\
  比例定数の次元は[J・s]である。Dirac定数を比例定数に用いると、$S = \hbar \phi$となる。\\
  これにより、光学から力学への類比が得られる。\\
  幾何光学が波動光学に移行する時の表現が、この関係を用いて古典力学から量子力学に移行する時の表現となる。\\
  波動関数は、位相$\phi$を用いて、
  \eq{
    \psi = A e^{i\phi}
  }
  と表される。\\
  これに作用$S$を用いると、
  \eq{
    \psi = A e^{\f{i}{\hbar}S}\label{eq:wavefunctionAction}
  }
  となる。これが量子力学における波動関数の基本的な表現である。\\
  注意したいのは、量子力学において粒子は決して確定した軌道を通らないことである。\\
  これが、確定した軌道を通る古典的極限にどう移行するのかを考える。\\
  Schrödinger方程式に式\eqref{eq:wavefunctionAction}を代入してみると、以下の2式が得られる。
  \eq{
    \pardiff{S}{t} + \f{(\nabla S)^2}{2m} + V -\f{\hbar ^2}{2mA} \Delta A &= 0 \label{eq:HJwithQuantumCorrection}\\
    \pardiff{A}{t} + \f{a}{2m}\Delta S + \f{1}{m}\nabla S \cdot \nabla A &= 0\label{eq:continuityEquation}
  }
  式\eqref{eq:HJwithQuantumCorrection}は、ハミルトン・ヤコビ方程式に量子補正項が加わったものである。\\
  この量子補正項は、$\hbar \to 0$で消失する。\\
  つまり、$\hbar \to 0$の極限では、式\eqref{eq:HJwithQuantumCorrection}はハミルトン・ヤコビ方程式になる。\\
  また、式\eqref{eq:continuityEquation}は変形すると、
  \eq{
    \pardiff{A^2}{t} + div(A^2 \f{\nabla S}{m}) = 0
  }
  となる。$\f{\nabla S}{m}$は古典力学における粒子の速度であるから、まさに$A^2$に関する連続の方程式である。(これに関しては$\hbar \to 0$をとっていないので量子力学の範疇でも確率密度が古典的速度に従って、古典力学の法則によって移動することがわかる。)\\
  